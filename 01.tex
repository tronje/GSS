\newcommand{\todo}{\textbf{\textcolor{red}{TODO}}}

\documentclass[a4paper,11pt]{scrartcl}
\usepackage[german,ngerman]{babel}
\usepackage[utf8]{inputenc}
\usepackage[T1]{fontenc}
\usepackage{lmodern}
\usepackage{enumerate}
\usepackage{fancyhdr}
\author{Moritz Mönnich, Rafael Epplee, Tronje Krabbe}
\title{GSS Hausaufgaben zum 16.04.}
\date{\today}

\pagestyle{fancy}
\fancyhf{}
\fancyhead[R]{Moritz M"onnich, Rafael Epplee, Tronje Krabbe}
\fancyhead[L]{GSS Hausaufgaben}
\fancyfoot[C]{\thepage}

\begin{document}
    \maketitle
    \textbf{Aufgabe 3: Angreifermodell} \\
            \begin{enumerate}
                \item[\textbf{1.)}]
                    Das Angreifermodell ist ein theoretisches Konstrukt, welches den schlimmst-m"oglichsten Angreifer beschreibt.
                    Will heißen, der st"arkste Angreifer, der noch von einem Schutzsystem abgewehrt wird. Es wird aufgestellt,
                    um einen "Uberblick zu erhalten, wie effektiv das Schutzsystem ist, und wo z.B. noch Verbesserungsbedarf besteht.
                    Ein Angreifermodell ber"ucksichtigt die folgenden Kriterien: \\
                    \textbf{Rollen:} Umfasst viele verschiedene Rollen, die kompakt in Insider und Outsider unterteilt werden k"onnen.\\
                    \textbf{Verbreitung:} Beschreibt die Stellen im System, auf die der Angreifer Zugriff hat.\\
                    \textbf{Verhalten:} Entweder aktiv oder passiv.\\
                    \textbf{Rechenkapazit"at:} Entweder beschr"ankt oder unbeschr"ankt.
            \end{enumerate}
\end{document}