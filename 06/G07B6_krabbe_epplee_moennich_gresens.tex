% "strict" mode, meckert bei kleineren fehlern
\RequirePackage[l2tabu, orthodox]{nag}

\documentclass[ngerman]{fbi-aufgabenblatt}
\usepackage{microtype}

% Folgende Angaben bitte Anpassen !!!

\renewcommand{\Aufgabenblatt}{6}
\renewcommand{\Gruppe}{G07}
\renewcommand{\KleinGruppe}{A}
\renewcommand{\Teilnehmer}{Krabbe, Flunder, Epplee, M�nnich, Gresens}

\begin{document}
\emph{Disclaimer: Leider hatten wir diese Woche nicht besonders viel Zeit und haben daher nur einen kleinen Teil der Aufgaben fertig bekommen. Sorry!}

\section{Speicherverwaltung}
    \subsection{}
        Bei einer virtuellen Adressgr��e von 16 Bit ergibt sich ein virtueller Adressraum mit der Gr��e $2^16 = 65536$ Bit bzw. 8192 Byte. Da es 16 Seiten gibt, folgt daraus eine Seitengr��e von $\frac{8192}{16} = 512$ Bytes bzw. W�rtern.

    \subsection{}
        Das Present/Absent-Bit gibt an, ob eine Seite aktuell im Hauptspeicher geladen ist. Es k�nnen also maximal so viele Seiten geladen sein, wie Platz im Hauptspeicher ist.
        Da die L�nge der physikalischen Adressen 15 Bit ist, kann man von einer Speichergr��e von $2^{15}$, also 32768 Bit oder 4096 Byte ausgehen. Bei einer Seitengr��e von 512 Bytes passen somit $\frac{4096}{512} = 8$ Seiten in den Hauptspeicher und es k�nnen maximal 8 Present/Absent-Bits gleichzeitig auf 1 gesetzt sein.
    \subsection{}
    \subsection{}
        F�r kleinere Seiten spricht, dass bei speicherarmen Prozessen nicht viel Speicher f�r leere Seiten vergeudet wird. Ebenso bieten kleinere Seiten mehr Flexibilit�t bei der Seitenverdr�ngung: Wenn nur ein kleiner Teil Speicher gebraucht wird, m�ssen auch nur wenige Daten auf die Platte geschrieben werden.

        F�r gro�e Seiten bzw. gegen kleinere Seiten spricht der Overhead, den eine kleine Seitengr��e verursacht: zum einen braucht man dann gr��ere Seitentabellen und zum anderen muss man eventuell �fter Seiten im Speicher hin- und herschieben und hat eine h�here Fragmentierung des Speicherbereichs.

\end{document}
