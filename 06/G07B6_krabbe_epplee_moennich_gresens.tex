% "strict" mode, meckert bei kleineren fehlern
\RequirePackage[l2tabu, orthodox]{nag}

\documentclass[ngerman]{fbi-aufgabenblatt}
\usepackage{microtype}

% Folgende Angaben bitte Anpassen !!!

\renewcommand{\Aufgabenblatt}{6}
\renewcommand{\Gruppe}{G07}
\renewcommand{\KleinGruppe}{A}
\renewcommand{\Teilnehmer}{Krabbe, Flunder, Epplee, M�nnich, Gresens}

\begin{document}
\emph{Disclaimer: Leider hatten wir diese Woche nicht besonders viel Zeit und haben daher nur einen kleinen Teil der Aufgaben fertig bekommen. Sorry!}

\section{Speicherverwaltung}
    \subsection{}
        Bei einer virtuellen Adressgr��e von 16 Bit ergibt sich ein virtueller
        Adressraum mit der Gr��e $2^16 = 65536$ Bit. Mit der Word gr��e von 1
        Bryte ergibt es $65536$ Byte. Da es 16 Seiten gibt, folgt daraus eine
        Seitengr��e von $\frac{65536}{16} = 4096$ Bytes.

    \subsection{}
        Das Present/Absent-Bit gibt an, ob eine Seite aktuell im Hauptspeicher
        geladen ist. Es k�nnen also maximal so viele Seiten geladen sein, wie
        Platz im Hauptspeicher ist.
        Da die L�nge der physikalischen Adressen 15 Bit ist, kann man von einer
        Speichergr��e von $2^{15}*8$Bit, also 32768 Byte ausgehen. Bei einer
        Seitengr��e von 4096 Bytes passen somit $\frac{32768}{4096} = 8$ Seiten
        in den Hauptspeicher und es k�nnen maximal 8 Present/Absent-Bits
        gleichzeitig auf 1 gesetzt sein.
    \subsection{}
    0x5fe8: Ersten 4 Bit geben Adresse der Seite an.
    Pr�fung ob Seite Nr. 5 existiert und im Arbeitsspeicher liegt
    (Present/Absent-Bit gesetzt).
    Physikalsiche Adresse: 0x1fe8

    \subsection{}
    
        F�r kleinere Seiten spricht, dass bei speicherarmen Prozessen nicht viel Speicher f�r leere Seiten vergeudet wird. Ebenso bieten kleinere Seiten mehr Flexibilit�t bei der Seitenverdr�ngung: Wenn nur ein kleiner Teil Speicher gebraucht wird, m�ssen auch nur wenige Daten auf die Platte geschrieben werden.

        F�r gro�e Seiten bzw. gegen kleinere Seiten spricht der Overhead, den eine kleine Seitengr��e verursacht: zum einen braucht man dann gr��ere Seitentabellen und zum anderen muss man eventuell �fter Seiten im Speicher hin- und herschieben und hat eine h�here Fragmentierung des Speicherbereichs.
    \subsection{}
    p=4MB L=8 B s=?
    Seitengr��e: $\frac{p}{s} \cdot L$ 
    Verschwendung: $V(s) = \frac{p}{s} \cdot L +  \frac{s}{2}$
    $V'(s) = - \frac{p}{s^2} + \frac{1}{2} = 0$
    $\rightarrow s = \sqrt{2pL}$
    $= \sqrt{64 \cdot 2^{20}}B = 8 KiB$


\section{}


\section{Synchronisation}
    \subsection{}
    Semaphore w =1
    int Num = 0
    Semaphore Mutex =1
    process Writer:
        P(w)
        modifyData
        V(w)


    processReader
        P(Mutex)
        Num++
        if(Num == 1)
            P(w)

        readData

        if(Num == 1)
            V(w)
        Num --
        V(Mutex)
\end{document}
