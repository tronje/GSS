\documentclass[ngerman]{fbi-aufgabenblatt}

% Folgende Angaben bitte Anpassen !!!

\renewcommand{\Aufgabenblatt}{1}
\renewcommand{\Gruppe}{G01}
\renewcommand{\KleinGruppe}{A}
\renewcommand{\Teilnehmer}{Krabbe, Epplee, M�nnich}

\begin{document}

\section{Allgmeine Aussagen zur IT-Sicherheit}

\subsection{Verteilte Systeme und Sicherheit}

\subsection{Ursachen}

\subsection{Angriffsformen}

\section{Schutzziele}

\subsection{Abgrenzung I}

\subsection{Abgrenzung II}

\subsection{Techniken}

\section{Angreifemodell}

\subsection{Angreifermodell}

Unter einem Angreifermodell versteht man ein Modell mit Hilfe dessen man die Wirksamkeit eines Schutzmechanismus definieren kann. Abgebildet wird die Wirksamkeit auf die maximale St�rke eines Angreifers, vor dem der Schutz noch besteht.
Angreifermodelle werden aufgestellt um die Wirksamkeit aktueller Schutzmechanismen zu vergleichen und zu analysieren. Zudem k�nnen sie herangezogen werden, um neue Schutzmechanismen zu entwickeln.

Die maximale St�rke eines Angreifers wird in unterschiedliche Kriterien aufgeteilt. Die Rolle, Verbreitung, das Verhalten und die Rechnerkapazit�t eines Angreifers.

Die Rolle es Angreifers beschreibt wie viel Wissen und Zugriff dem Angreifer zur Verf�gung stehen, also ob dieser ein Insider (Produzent, Entwerfer...) oder Outsider(Benutzer...) ist.

Die Verbreitung des Angreifers beschreibt die Orte an denen Informationen von einem Angreifer abgegriffen werden k�nnen. Dies k�nnte als Beispiel das lokale Netzwerk oder alle in einem Land befindlichen Backbones sein.

Das Verhalten des Angreifers, sagt aus ob dieser passiv handelt, also nur Daten beobachten kann oder ob dieser aktiv in Systeme oder Datenfl�sse eingreift.
Ein passiver Angreifer k�nnte z.B jemand sein der Funksignale mith�rt oder aufzeichnet.

Die Rechenkapazit�t eines Angreifers ist bedeutend bei mathematischen bzw. kryptographischen Schutzmechanismen. Dessen Schutz basiert n�mlich auf Problemen mit hinreichend langer Rechenzeit. Ein Angreifer, der �ber gen�gend Rechenkapazit�t verf�gt kann diese Rechenzeit minimieren und somit den Schutz umgehen.
Dies ist au�erdem ein wichtiges Kriterium f�r die Schutzdauer eines Schutzmechanismus in Bezug auf die immer weiter zunehmende Rechenkapazit�ten.


\subsection{Praxisbeispiel}

\section{Passwortsicherheit}

\subsection{Einfaches Hash-Verfahren}

\subsection{Brute-Force-Angriff}

\subsection{Time-Memory-Trade-Off}

\subsection{Salting}

\subsection{Dictionary-Attack}


%\input{betriebssysteme}

\end{document}
