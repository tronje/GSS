\documentclass[ngerman]{fbi-aufgabenblatt}
% \usepackage{minted}

% Folgende Angaben bitte Anpassen !!!

\renewcommand{\Aufgabenblatt}{5}
\renewcommand{\Gruppe}{G07}
\renewcommand{\KleinGruppe}{A}
\renewcommand{\Teilnehmer}{Krabbe, Epplee, M�nnich, Gresens}

\begin{document}

\section{Zentrale Begriffe der Kryptographie}
	\subsection{Unterschiedliche Chiffren}
		\begin{itemize}
			\item Symmetrisches Kryptosystem
			\begin{itemize}
				\item Anz. d. Schl�ssel: 1
				\item Verwendung f�r Ver- und Entschl�sselung
				\item Alle involvierten Personen m�ssen den Schl�ssel geheimhalten
			\end{itemize}

			\item Asymmetrisches Kryptosystem
			\begin{itemize}
				\item Anz. d. Schl�ssel: 2
				\item �ffentlicher Schl�ssel f�r die Verschl�sselung
				\item Privater Schl�ssel f�r die Entschl�sselung
				\item Jede Person muss nur ihren eigenen privaten Schl�ssel geheimhalten
			\end{itemize}
		\end{itemize}
	\subsection{Hybride Kryptosysteme}
	\subsubsection*{a)}
	Je l�nger die Nachricht ist, desto uneffizienter ist die Ver- und Entschl�sselung, daher lohnt sich ab einer gewissen L�nge der Nachricht der Wechsel auf hybrides Kryptosystem.
	\subsubsection*{b)}
	Alice erzeugt einen neuen symmetrischen Session-Key k und verschl�sselt mit diesem die Nachricht an Bob.
	Dann verschl�sselt sie k assymmetrisch mit Bobs public key und schickt die verschl�sselte Nachricht und den verschl�sselten Schl�ssel an Bob.
	\subsubsection*{c)}
	Die Nachricht setzt sich aus den symmetrisch verschl�sselten Nutzdaten (der eigentlichen Nachricht) und dem assymmetrisch verschl�sselten Session-Key zusammen

\section{Parkhaus}
	\subsection{Funktionsweise}
	Die letzten 6 Ziffern des ersten Barcodes (von rechts) stehen umsotiert unten rechts: 1-> 5, 2-> 2, 3-> 6, 4-> 4, 5-> 1, 6-> 3
	
	\subsection{Sicherheitsanalyse}
	\subsection{Umsetzung mit kryptographischen Techniken}


\section{Authentifizierungsprotokolle}
	\subsection{Verschl�sselte Passwort-�bermittlung}
	\subsection{Authentifikationssystem auf Basis indeterministischer symmetrischer Verschl�sselung}
	\subsection{Challenge-Response-Authentifizierung}
	\subsection{Sichere Challenge-Response-Authentifizierung}

\section{"Mensch �rgere Dich nicht" �ber das Telefon}
	\subsection{Protokoll}
	\subsection{W�rfeln �ber Telefon}

\section{RSA-Verfahren}
	\subsection{Grundlagen}
	\subsection{Anwendung}
	\subsection{Sichere Implementierung}

\end{document}
