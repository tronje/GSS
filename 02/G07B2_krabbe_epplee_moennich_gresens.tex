\documentclass[ngerman]{fbi-aufgabenblatt}

% Folgende Angaben bitte Anpassen !!!

\renewcommand{\Aufgabenblatt}{2}
\renewcommand{\Gruppe}{G07}
\renewcommand{\KleinGruppe}{A}
\renewcommand{\Teilnehmer}{Krabbe, Epplee, M�nnich, Gresens}

\begin{document}

\section{}
    
\subsection{Grundlagen von Betriebssystemen}
\begin{itemize}
    \item a)
        \subitem Betriebsmittelverwalter: Betriebsmittel (Zeit und Speicher) von Prozessen werden verwaltet, Probleme und Konflitke sollen behandelt werden.
        \subitem Virtuelle Maschiene: Hardwaredetails werden vor dem Benutzer verborgen. Dies sorgt f�r mehr hardwareunabh�ngigkeit.
    \item b)
        \subitem Sicht 1:
        \subsubitem (1) Zuordnung von Betriebsmitteln zu einzelnen Prozessen
        \subsubitem (2) Behandlung von Problemen und Konflikten bei Inanspruchnahme von Betriebsmitteln, u.a. Behandlung von Verklemmungs Verklemmungs- und Engpass Engpass-Situationen
        \subitem Sicht 2:
        \subsubitem (1) Zugrunde liegende Rechnerarchitektur vor dem Nutzer verbergen
        \subsubitem (2) Details des Mehrbenutzerbetriebs verstecken
    \item c)
        \subitem monolisthisch
            \subsubitem schnellerer Zugriff
            \subsubitem un�bersichtlich unstrukturiert
        \subitem hirachisch
            \subsubitem strukturierter
            \subsubitem sicherer
            \subsubitem nicht so gute perfomance
    \item d)
        \subitem Netwerk
            \subsubtitem hohe Verteilung-Transparenz
        \subitem Verteilt
            \subsubtitem keine/geringe Verteilungs-Transparenz
\end{itemize}
\subsection{Prozesse und Threads}
\begin{itemize}
    \item a)
        \subitem Ein Programm ist eine Folge von Anweisungen und hat eine bestimmte Ziel-Pragmatik. Otfmals in einer bestimmten Programmiersprache geschrieben und in f�r einen Computer verst�ndliche Befehle �bersetzt.
        \subitem Als Prozess wird ein Programm bezeichnet, das aktuell auf dem Betriebssystem ausgef�hrt wird. Besitzt einen eigenen Prozessadressraum der u.a. Programmcode, Konstanten und prozessspezifische Variablen enth�lt.
        \subitem Die eigentliche Ausf�hrung eines Prozess wird in kleineren Mini-Prozesse, s.g. Threads statt. Alle Threads eines Prozess teilen sich den gleichen Prozessadressraum, besitzen jedoch wegen ihrer Nebenl�ufigkeit einen eigenen Stack.
    \item b) 
        \subitem Zufgriff auf selbe Betriebsmittelk�nnen zu komplikationen f�hren.
        \subitem => Deadlock
    \item c)
        \subitem z.B Benutzeroberfl�che aktualisieren
        \subitem k�nnen auf den selben Speicher zugreifen und so einfacher kommunizieren
        \subitem Ressourcen sparender
    \item d)
        \subitem X:= BEREIT: Der Prozess besitzt alle ben�tigten Betriebsmittel, wartet jedoch auf die Freigabe eines Prozessorkerns.
        \subitem Y:= LAUFEND: Der Prozess ist aktuell einem Prozessorkern zugeordnet und l�uft auf diesem ab.
        \subitem Z:= WARTEND: Der Prozess wurde unterbrochen und wartet auf eine nicht-Prozessorkern-Ressource.
        \subitem a:= (NEW nach X) Starten des Prozess \& Laden aller ben�tigten Betriebsmittel
        \subitem b:= (X nach Y) Zuweisung von Prozessorkern(en)
        \subitem c:= (Y nach Z) Ben�tigung eines momentan nicht verf�gbaren Betriebsmittels (I/O)
        \subitem d:= (Z nach X) Zuteilung des Betriebsmittel
        \subitem e:= (Y nach TERMINATED) Beenden des Prozess, Berechnung ist vollst�ndig
        \subitem f:= (Y nach X) Pausieren des Prozess
\end{itemize}
\subsection{n-Adressmaschine}
\begin{itemize}
    \item a) 2-Adress Maschine: 
        Leseauftr�ge: 15, Schreibauftr�ge: 10 Berechnungszeit: 25,5
        \begin{verbatim}
        MOVE >a1< >H1<  H1:=a1
        ADD  >a2< >H1<  H1:=a2 + H1
        MOVE >a3> >H2<  H2:=a3
        DIV  >H1< >H2<  H2:=H1 / H2
        MOVE >b2< >H3<  H3:=b2
        SUB  >b1< >H3<  H3:=b1 - H3
        MOVE >b3> >H4<  H4:=b3
        DIV  >H3< >H4<  H4:=H3 / H4
        ADD  >H2< >H4<  H4:=H2 + H4
        MOVE >H4< >R<   R:=H4
        \end{verbatim}
    \item b) 1-Adress-Maschine:
        Leseauftr�ge: 9, Schreibauftr�ge: 4 Berechnungszeit: 13,65
        Vertauscht!:
        \begin{verbatim}
        LOAD >b2<      AC:=b2           AC=b2
        SUB  >b1<      AC:=b2 - AC      AC=b1 - b2
        DIV  >b3<      AC:=AC / b3      AC=(b1 - b2) / b3
        SAVE >H1<      H1:=AC       
        LOAD >a1<      AC:=a1           AC=a1
        ADD  >a2<      AC:=AC + a2      AC=a1 + a2
        DIV  >a3<      AC:=AC / a3      AC=(a1 + a2) / a3
        ADD  >H1<      AC:=AC + H1      AC=((a1 + a2) / a3) + ((b1 - b2) / b3)
        SAVE >R<       R:=AC        
        \end{verbatim}
    \item c) 0-Adress-Maschine:
        Leseauftr�ge: 6, Schreibauftr�ge: 1,  Berechnungszeit: 7,6
        \begin{verbatim}
        PUSH >a1<   ;a1;
        PUSH >a2<   ;a1;a2;
        ADD         :a1 + a2;
        PUSH >a3<   ;a1 + a2;a3;
        DIV         ;(a1 + a2) / a3;
        PUSH >b1<   ;(a1 + a2) / a3;b1;
        PUSH >b2<   ;(a1 + a2) / a3;b1;b2;
        SUB         ;(a1 + a2) / a3;b1 - b2;
        PUSH >b3<   ;(a1 + a2) / a3;b1 - b2;b3;
        DIV         ;(a1 + a2) / a3;(b1 - b2) / b3;
        ADD         ;((a1 + a2) / a3) + ((b1 - b2) / b3);
        POP  >R<    ;
        \end{verbatim}
\end{itemize}
\end{document}
