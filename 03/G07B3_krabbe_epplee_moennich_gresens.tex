\documentclass[ngerman]{fbi-aufgabenblatt}
\usepackage{minted}

% Folgende Angaben bitte Anpassen !!!

\renewcommand{\Aufgabenblatt}{3}
\renewcommand{\Gruppe}{G07}
\renewcommand{\KleinGruppe}{A}
\renewcommand{\Teilnehmer}{Krabbe, Epplee, M�nnich, Gresens}

\begin{document}
\section{Rechnersicherheit}
    \subsection{Zugangs- und Zugriffskontrolle}
        \begin{itemize}
            \item a)
                \subitem Zungangskontrolle: Beschr�nkt, wer �berhaupt Zugang zu dem System hat. Z.B.: Username/Passwort-Abfrage auf bei der Anmeldung auf einem Server.
                \subitem Zugriffskontrolle: Beschr�nkt, auf was genau im System der User Zugriff hat, nachdem ihm Zugang gew�hrt wurde.
            \item b)
                Grunds�tzlich ist diese M�glichkeit nicht auszuschlie�en. Beispielsweise in Systemen mit wenigen oder einzig und allein vertrauensw�rdigen Nutzern, oder in Systemen, in denen Benutzer einfach grunds�tzlich nichts schlimmes anstellen k�nnen.
            \item c)
                Man kann die Zugriffsrechte eines Users nicht �berpr�fen/durchsetzen, wenn man ihn nicht vorher identifiziert hat. Um einen User zu identifizieren, ben�tigt man mindestens Zugangskontrolle.
            \item d)
                Es wird hier von einer Zugriffskontrolle auf \textit{Ordner-Ebene} geredet. Das bedeutet, dass f�r bestimmte Ordner die Zugriffskontrolle \textit{aufgehoben} wird. Diese Ordner sind dann �ffentlich zug�nglich. Ordner, die nicht freigegeben sind, sind weiterhin nur f�r eingeloggte Benutzer sichtbar, die Zugangskontrolle ist also weiterhin aktiv und es gilt das Statement aus aufgabe c).
        \end{itemize}

    \subsection{Biometrische Techniken: EasyPASS}
        \begin{itemize}
            \item a)
            \item b)
                \subitem Webcam: kann manipuliert werden und fremdes Bild �bermitteln.
        \end{itemize}

    \subsection{Tippverhalten}
        \begin{itemize}
            \item a) Der Angreifer kann den Benutzer dazu bringen den
                Authentifizierungs-Satz auf einer eigenen Website einzugeben, um
                so das Tippverhalten mitzuschneiden. Dies kann er dann benutzen
                um sich zu authentifizieren. 
            \item b) Eine Gegenma�nahme k�nnte sein f�r jeden Benutzer einen
                zuf�lligen Authentifizierungs-Satz zu w�hlen, der dem Angreifer
                nicht bekannt ist.
        \end{itemize}

        \subsection{Realisierung eines Online Tickets}
        \begin{itemize}
            \item
        \end{itemize}

\section{Timing-Attack}
    \begin{itemize}
        \item 1.
            \begin{minted}{java}
                boolean isTimingAttackPossible()
                {
                    char[] pass1 = ['h','a','s','e'];
                    char[] pass2 = ['t','i','g','e','r','e','n','t','e'];

                    int t1 = System.nanoTime();
                    passwordCompare(pass1, pass1);
                    int t2 = System.nanoTime();

                    int result1 = math.abs(t1 - t2);

                    t1 = System.nanoTime();
                    passwordCompare(pass1, pass2);
                    t2 = System.nanoTime();

                    int result2 = math.abs(t1 - t2);

                    return result1 == result2;
                }
            \end{minted}

            Man k�nnte beide Varianten von passwordCompare zwischen den Zeitmessungen mehrfach ausf�hren, um die Zeitunterschiede deutlicher zu machen.
        \item 2.
            Eine Timing-Attack ist hier m�glich, da in passwordCompare entweder
            sofort abgebrochen, oder eine naive for-Schleife ausgef�hrt wird,
            welche stoppt sobald ein Zeichen nicht �bereinstimmt.
            Dies kann zu sehr unterschiedlichen Ausf�hrungszeiten f�hren.         
        \item 3.
            Zun�chst versucht der Angreifer, die L�nge des Passworts herauszufinden. Hierf�r probiert er der Reihe nach Passw�rter verschiedener L�nge (und mit beliebigem Inhalt) aus: \texttt{a, aa, aaa, aaaa} etc. Dabei sollten alle Passw�rter bis auf eines gleich lang brauchen. Das eine Passwort, das l�nger braucht (da das Programm nicht sofort abbricht, sondern beginnt, die Passw�rter Zeichen f�r Zeichen zu vergleichen), hat die richtige L�nge.

            Nun beginnt der Angreifer, das erste Zeichen des Passwortes zu knacken. Hierf�r probiert er Passw�rter mit der richtigen L�nge aus und �ndert dabei immer das erste Zeichen, z.B. (bei einer L�nge von 4) \texttt{aaaa, baaa, caaa, daaa} etc. Auch hier wird wieder ein Passwort l�nger brauchen als die anderen, da das Programm das erste Zeichen dieses Passwortes als richtig ansieht und noch (mindestens) das zweite Zeichen vergleicht. So kann der Angreifer sicher sein, dass das erste Zeichen dieses Passwortes das richtige ist und wendet nun die gleiche Methode auf das zweite Zeichen an. Nachdem er jedes Zeichen geknackt hat, besitzt der Angreifer das korrekte Passwort.
            
        \item 4.
            \begin{minted}{java}
                // b ist die Benutzereingabe!
                boolean passwordCompare(char[] a, char[] b) 
                {
                    int e;
                    int timefake;
                    boolean correctLength = if(a.length == b.length);
                    for (i=0; i < a.length; i++)
                    {
                        if (correctLength && a[i] == b[i])
                        {
                            e++;
                        } elseif (a[i] == a[i]) 
                        {
                            timefake++;
                        }
                    }
                    return  i == a.length ;

                \end{minted}

    \end{itemize}
\end{document}
