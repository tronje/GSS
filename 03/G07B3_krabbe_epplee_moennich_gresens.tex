\documentclass[ngerman]{fbi-aufgabenblatt}
\usepackage{minted}

% Folgende Angaben bitte Anpassen !!!

\renewcommand{\Aufgabenblatt}{3}
\renewcommand{\Gruppe}{G07}
\renewcommand{\KleinGruppe}{A}
\renewcommand{\Teilnehmer}{Krabbe, Epplee, M�nnich, Gresens}

\begin{document}
\section{Rechnersicherheit}
    \subsection{Zugangs- und Zugriffskontrolle}
        \begin{itemize}
            \item a)
                \subitem Zungangskontrolle: Beschr�nkt, wer �berhaupt Zugang zu dem System hat. Z.B.: Username/Passwort-Abfrage auf bei der Anmeldung auf einem Server.
                \subitem Zugriffskontrolle: Beschr�nkt, auf was genau im System der User Zugriff hat, nachdem ihm Zugang gew�hrt wurde.
            \item b)
                Grunds�tzlich ist diese M�glichkeit nicht auszuschlie�en. Beispielsweise in Systemen mit wenigen oder einzig und allein vertrauensw�rdigen Nutzern, oder in Systemen, in denen Benutzer einfach grunds�tzlich nichts schlimmes anstellen k�nnen.
            \item c)
                Man kann die Zugriffsrechte eines Users nicht �berpr�fen/durchsetzen, wenn man ihn nicht vorher identifiziert hat. Um einen User zu identifizieren, ben�tigt man mindestens Zugangskontrolle.
            \item d)
                Es wird hier von einer Zugriffskontrolle auf \textit{Order-Ebene} geredet. Das bedeutet, dass f�r bestimmte Ordner die Zugriffskontrolle \textit{aufgehoben} wird. Diese Ordner sind dann �ffentlich zug�nglich. Ordner, die nicht freigegeben sind, sind weiterhin nur f�r eingeloggte Benutzer sichtbar, die Zugangskontrolle ist also weiterhin aktiv und es gilt das Statement aus aufgabe c).
        \end{itemize}

\section{Timing-Attack}
    \begin{itemize}
        \item 1.
            \begin{minted}{java}
                boolean isTimingAttackPossible()
                {
                    char[] pass1 = ['h','a','s','e'];
                    char[] pass2 = ['t','i','g','e','r','e','n','t','e'];

                    int t1 = System.nanoTime();
                    passwordCompare(pass1, pass1);
                    int t2 = System.nanoTime();

                    int result1 = math.abs(t1 - t2);

                    t1 = System.nanoTime();
                    passwordCompare(pass1, pass2);
                    t2 = System.nanoTime();

                    int result2 = math.abs(t1 - t2);

                    return result1 == result2;
                }
            \end{minted}

            Man k�nnte beide Varianten von passwordCompare zwischen den Zeitmessungen mehrfach ausf�hren, um die Zeitunterschiede deutlicher zu machen.
        \item 2.
        \item 3.
            Zun�chst versucht der Angreifer, die L�nge des Passworts herauszufinden. 
        \item 4.
    \end{itemize}
\end{document}
