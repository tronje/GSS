\documentclass[ngerman]{fbi-aufgabenblatt}

% Folgende Angaben bitte Anpassen !!!

\renewcommand{\Aufgabenblatt}{1}
\renewcommand{\Gruppe}{G07}
\renewcommand{\KleinGruppe}{A}
\renewcommand{\Teilnehmer}{Krabbe, Epplee, M�nnich}

\begin{document}

\section{Allgmeine Aussagen zur IT-Sicherheit}
    
\subsection{Verteilte Systeme und Sicherheit}
\begin{itemize}
    \item Vorteile
        \subitem Redundanz
        \subitem Verf�gbarkeit
        \subitem Datenverlust
    \item Nachteile
        \subitem Mehr Angriffsvektoren (z.B Kommunikationswege)
        \subitem Mehr Aufwand durch Schutz
        \subitem Komplexer
\end{itemize}
\subsection{Ursachen}
    Zeitmangel
    Kompetenz
    Sicherheitsbewusstsein
    Geld
\subsection{Angriffsformen}
    
\section{Schutzziele}
\begin{itemize} 
    \item a)
        \subitem Verkehrsfulussanalyse
        \subitem passive Beobachtung
    \item b)
        \subitem Man in the Middle
            \subsubitem Verf�gbarkeit
            \subsubitem Integrit�t
            \subsubitem Vertraulichkeit
        \subitem Schutz:
            \subsubitem VPN
            \subsubitem SSl/TLS
            \subsubitem automatisches Verbinden deaktivieren
\end{itemize}
\subsection{Abgrenzung I}
\begin{itemize}
    \item Anonymit�t
        \subitem Identit�t kann nicht zur�ckverfolgt werden
    \item Pseudonymit�t
        \subitem Aktionen lassen sich verketten, Identit�t l�sst sich zur�ckverfolgen 
    \item Unbeobachtbarkeit
        \subitem Es kann nicht ermittelt werden, dass ein Teilnehmer kommuniziert. 
\end{itemize}
\subsection{Abgrenzung II}
\begin{itemize}
    \item Vertraulichkeit
        \subitem Nachrichten Inhalt kann nicht gelesen werden
        \subitem Kryptographie
    \item Verdecktheit
        \subitem Existenz der Nachricht verborgen
        \subitem Steganographie


    \item Integrit�t
        \subitem festellten einer Modifikation
    \item Zurechenbarkeit
        \subitem Uhrheber festellbar

    \item Erreichbark keit
        \subitem Auf das System kann zugegruffen werden
    \item Verf�gbarkeit
        \subitem Dienste laufen zuverl�ssig
\end{itemize}

\subsection{Techniken}

\begin{itemize}
    \item Onion-Network (Tor)
    \item Dummy-Nachrichten
    \item Verschl�sselung
    \item Steganographi
    \item Hashes MAC
    \item Signaturen
    \item Redundanz/ Diversit�t
    \item verschieden Implementierungen

\end{itemize}

\section{Angreifemodell}

\subsection{Angreifermodell}

Unter einem Angreifermodell versteht man ein Modell mit Hilfe dessen man die Wirksamkeit eines Schutzmechanismus definieren kann. Abgebildet wird die Wirksamkeit auf die maximale St�rke eines Angreifers, vor dem der Schutz noch besteht.
Angreifermodelle werden aufgestellt um die Wirksamkeit aktueller Schutzmechanismen zu vergleichen und zu analysieren. Zudem k�nnen sie herangezogen werden, um neue Schutzmechanismen zu entwickeln.

Die maximale St�rke eines Angreifers wird in unterschiedliche Kriterien aufgeteilt. Die Rolle, Verbreitung, das Verhalten und die Rechnerkapazit�t eines Angreifers.

Die Rolle es Angreifers beschreibt wie viel Wissen und Zugriff dem Angreifer zur Verf�gung stehen, also ob dieser ein Insider (Produzent, Entwerfer...) oder Outsider(Benutzer...) ist.

Die Verbreitung des Angreifers beschreibt die Orte an denen Informationen von einem Angreifer abgegriffen werden k�nnen. Dies k�nnte als Beispiel das lokale Netzwerk oder alle in einem Land befindlichen Backbones sein.

Das Verhalten des Angreifers, sagt aus ob dieser passiv handelt, also nur Daten beobachten kann oder ob dieser aktiv in Systeme oder Datenfl�sse eingreift.
Ein passiver Angreifer k�nnte z.B jemand sein der Funksignale mith�rt oder aufzeichnet.

Die Rechenkapazit�t eines Angreifers ist bedeutend bei mathematischen bzw. kryptographischen Schutzmechanismen. Dessen Schutz basiert n�mlich auf Problemen mit hinreichend langer Rechenzeit. Ein Angreifer, der �ber gen�gend Rechenkapazit�t verf�gt kann diese Rechenzeit minimieren und somit den Schutz umgehen.
Dies ist au�erdem ein wichtiges Kriterium f�r die Schutzdauer eines Schutzmechanismus in Bezug auf die immer weiter zunehmende Rechenkapazit�ten.


\subsection{Praxisbeispiel}
\begin{itemize}
    \item Rolle:
        \subitem Au�enstehender
    \item Verbreitung:
    \item Verhalten:
        \subitem Beobachtend
    \item Rechenkapazit�t
\end{itemize}
\section{Passwortsicherheit}

\subsection{Einfaches Hash-Verfahren}
    Sicherheit der Benutzerpassw�rter, trotz komprimitierung
    Passwort varibel lang, Hash immer gleich lang, effektievere Passwortverwaltung

    Kollisionsresistenz
    sollte nur schwer Invertriebar sein (Einwegfunktion)
\subsection{Brute-Force-Angriff}
   $62^8 + 62+7 +62^6 + 62^5 .... $
\subsection{Time-Memory-Trade-Off}
\begin{itemize}
    \item Bruteforce
        \subitem kein Speicherbedarf
        \subitem hoher linearer Zeitaufwand

    \item Alternative: Datenbank
        \subitem Zeitaufwand gering
        \subitem viel Speicherbedarf

    \item Kompromiss: Rainbow Table
\end{itemize}
        
\subsection{Salting}
\begin{itemize}
    \item Schutz gegen Rainbow Tables
    \item h(Salt|Password)
    \item f�r jedes Passwort neuer Salt
    \item Speicherung Hash + Salt
\end{itemize}
\subsection{Dictionary-Attack}
    

%\input{betriebssysteme}

\end{document}
